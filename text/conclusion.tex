\chapter{Conclusion}
\label{ch:conclusion}
%The proposed system is a standalone password manager equipped with the
%USB device. The device is hard to be tampered and is securely to store
%the pair (e.g., key:password) of user's credentials, where the key is
%a unique identity requiring to match the corresponding password
%(e.g. website, hostname, username, \dots). The application of the USB
%device provides an interface for a user to access the encrypted
%partition in the USB device.  The credentials that a user wants to
%store are located in the encrypted partition of the USB device which
%can be decrypted only with the primary password that a user setup.
%The default password is provided for the user to decrypt the encrypted
%partition for the first usage. The application forces a user to update
%the default password with a different password.  The USB device is
%updated with the new release of the application which can be
%downloaded from the designated webserver. The webserver uses the two
%factor authentication to ensure the authentictiy and integrity of the
%applications and the USB device. The user is required to register to
%the website with the public key of the USB device.  The new
%application is digitally signed with the factory's public key. Once
%the new application is verified to be authentic, it replaces the
%existing version in the USB device and restarts the application to
%load up with the new version.

%We have setup a threat model for the proposed system by using the
%standard STRIDE threat.  Our threat model is based on the iterative
%design. We review our proposed system architecture, dataflows, and
%external entities. Any potential threats are identified by using the
%STRIDE methodology and augmented by experimenting with the elevation
%of privilege game cards.  We discuss any identified trheats and assign
%a bug number to each one. Finally we decide whether we accept it as a
%system risk or not and perfom a risk likelihood vs impact for all
%accepted risks using a modified OWASP risk methodology.

Our threat model identified 43 threats and 19 risks (see
Figure~\ref{fig:riskmatrix}) associated with the system threats. Of
the 19 identified risks, 10 were classified as high risks largely
because any compromise of the system password led to complete loss of
confidentiality.  On the other hand, the likelihood of the threats is
4.05 in average within 1 to 10 range and the 10 high risks has 4.56. A
prinicpal attack vector was through the webserver using a malware
infected software/firmware image that is passed to an unaware
device. This attack tree is described in \nameref{ch:a4}. The risk of
this threat is also high category 2.

While the system appears to have significant vulnerabilities, we
believe that the likelihood element of the risk severity is overstated
in part because the OWASP risk methodology used does not actually
consider likelihood in the in light of design and mitigations. A
lesson learned that would apply to an even more comprehensive threat
model would be to iterate against mitigations which would reduce many
of the high risk threats to lower severity. If a compromise were to
occur, the impact is quite severe for most risks.  Finally, we do
transfer one risk to the user, that of properly protecting the USB
device and using it only on machines determined to be safe and free
from compromise.
