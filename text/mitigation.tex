\chapter{Mitigations}
\label{ch:mitigations}
In this chapter we describe the mitigations to the threats we found in the previous chapter.

\section{Risk and Prioritization}
\label{sec:risk}
Here we describe the risk management methods for our mitigations. I suggest we consider a likelihood and impact methodology and deliberately break out categorical threats and use a BUGBAR such as deploy with no CAT1 findings.
\marginnote{The style I am using puts notes in the margin to emphasize particular elements in a chapter or section.  In this case I would emphasize what a CATEGORY 1 finding is and refer the reader to an appendix for definition}

\subsection{Risk Definitions}
Here we place the definitions of our risk to include definitions of each type of likelihood and severity.  We then describe the risks using the definitions of major category such as CAT-1.  We might be able to utilize the DOD DIACAP\footnote{The DIACAP process can be found at \url{www.prim.osd.mil/Documents/diacap_workflow_map.pdf}} severity codes.  (Although I hate DIACAP personally)
