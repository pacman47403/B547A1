\chapter{System Overview}
\label{ch:System Overview}
The system modeled is a web-enabled password manager that resides on a
tamper resistant USB thumb drive, modeled after the
IronKey\texttrademark secure USB thumb drive. The password manager
allows for a user to enjoy the benefits of a secure offline password
manager with the option of receiving software and firmware updates from the
cloud. The USB device also hosts an embedded crypto system which can be
leveraged to provide powerful cryptographic tools which providing the assurance
that the tools have not been compromised.

\begin{marginfigure}%
\centering
  \includegraphics[width=0.25\linewidth]{s1000-vertical}
  \caption{Picture of the IronKey USB drive.  More information can be
found at \url{www.ironkey.com}}
  \label{fig:ik}
\end{marginfigure}



\section{System Description}
\label{sec:sysdesc}

The password manager consists of the USB drive, a webserver for software and
firmware updates and an embedded password manager app.  The password manager is
accessed through the USB drive which must first be unlocked via a host machine
running Windows, Mac or Linux operating systems. The USB drive contains an
encrypted partition and an unencrypted partition.  The encrypted partition is
formed on USB initialization and uses a unique random key inside of the USB to
generate a second random key which is used to create the encrypted
partition. These two keys are embedded in the hardware and never shared with the
host OS.\sidenote{The device unique identifier and private key are contained
within the silicon of the device and can only be accesed through an
authenticated secure session application hosted on the USB drive.}  The USB also
provides cryptographic functions via a cryptochip inside the USB.  The USB also
provides a unique identifier which is contained within its embedded circuitry.
\par Once the USB drive is inserted into the host computer, a program is
executed from the unencrypted partition which mounts the USB to the host OS.
Following mounting, a small program from the unencrypted partition is executed
which prompts the user for a password. After a preset number of failed password
attempts, the USB device will wipe all keys and execute additional functions to
render the USB device useless and unrecoverable.
\par The USB device firmware and software can be updated via a secure Internet
channel which accesses a webserver to download the device files to the USB
device.  Following download, the device authenticates the downloaded files and
updates the device.
The way to save the password via the password manager is straightforward. The
password is passed along with the corresponding key (e.g. a website, a username,
hostname, \dots) to the password manager when the secure user session is
setup as shown in Figures~\ref{fig:dfd_app} and ~\ref{fig:dfd_usb}. The key
must be unique. Once
the password is stored with the corresponding key to the encrypted partition in
the USB, a user can retrieve the password with the unique given key during the
secure user session. The retrieved password can be blinded out and copied to the
temporary memory of the system (e.g. clipboard) for users' convenience. Of
course, if a user inquires, the password can be displayed in a plain text but
preferably in an image format.

\begin{figure}
    \centering
    \includegraphics{DFD_Main}
    \caption{Main System Data Flow Diagram}
    \label{fig:dfd_main}
\end{figure}
\begin{figure*}
    \centering
    \includegraphics[width=\linewidth]{DFD_App}
    \caption{System Data Flow Diagram in the App side}
    \label{fig:dfd_app}
\end{figure*}

\begin{figure*}
    \centering
    \includegraphics[width=\linewidth]{DFD_USB}
    \caption{System Data Flow Diagram in the USB side}
    \label{fig:dfd_usb}
\end{figure*}

\begin{figure*}
    \centering
    \includegraphics[width=0.8\linewidth]{webservdfd1}
    \caption{System Data Flow Diagram for the Webserver}
    \label{fig:wsdfd1}
\end{figure*}


\begin{figure*}
    \centering
    \includegraphics[width=0.8\linewidth]{webservdfd2}
    \caption{System Data Flow Diagram for the Webserverer}
    \label{fig:wsdfd2}
\end{figure*}

The Password manager consists of two major components, the USB device and a User
Application. The USB device when plugged in the system is loaded as a CD
ROM. The Get/Load function from the application gets the public key and begins
the message
encryption. Once the user application is authenticated and loaded there can be 4
possibilities:
\begin{enumerate}
\item The password manager is used for the first time by the user
\item Correct credentials are entered by the user
\item Incorrect credentials are entered by the user
\item User enters incorrect credentials multiple times
\end{enumerate}

When the user uses the iron key for the first time, the header is decrypted with
the default key. Once the header decrypt is successful the user is forced to
reset the default password. When correct credentials are entered by the user,
the password key is sent to the USB which then decrypts the header with that
key.  Once the header is decrypted a a user session is started. The counter
which stores the number of invalid attempts is reset.The user has the option to
update the user password. The decrypted file is read into the memory and a new
password key and password file/header is written to the USB.A pair of key and
password is stored in the USB device. The password is queried from the USB
device using the given key.

If the user enters an invalid credentials, the header decrypt fails and the
password try counter is incremented.\marginnote{The USB itself does not possess
any logging capability save for the try counter which also serves as an
initiator to clear the system after a number of incorrect tries.} The user is
then prompted to enter a correct password. When a user enters a wrong password
multiple times which is equal to some threshold value, the reset counter
overwrites the header and password file by default.

The user application can be updated. When the update is successful, the USB is
updated and a force restart happens with the new version.When an update fails
the current version of the app is maintained.

The system data flow diagram is shown in figure \ref{fig:dfd_main}.


\section{Functional Requirements}
\label{sec:funcreq}
The functional requirements of the system are as follows:
\begin{enumerate}
    \item{Physically protect the USB against known tamper attacks including
physical attacks and USB electrical channel attacks.}\sidenote{We do not
enumerate all the possible USB physical and electrical attacks, nor do we
describe in detail the mitigations to these attacks.  It is assumed that
appropriate hardware antitamper features are incorporated into the USB design.}
    \item{Host all cryptography on the USB device.}
    \item{Allow for remote updates of USB software and firmware.}
    \item{Initiate device wipe after a set number of incorrect login attempts.}
    \item{Allow for the secure storage and retrieval of user websites and login
information.}
    \item{Provide for a secure, random password generator capability.}
\end{enumerate}
