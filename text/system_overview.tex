\chapter{System Overview}
\label{ch:System Overview}
The system modeled is a web-enabled password manager that resides on a
tamper resistant USB thumb drive, modelled after the
IronKey\texttrademark secure USB thumb drive. The password manager
allows for a user to enjoy the benefits of a secure offline password
manager with the option of storing information to the cloud.
Additionally, the benefits of a embedded crypto system provide
robustness to the user.  The device firmware can be updated via an
internet connection.

\begin{marginfigure}%
\centering
  \includegraphics[width=0.25\linewidth]{s1000-vertical}
  \caption{Picture of the IronKey USB drive.  More information can be
found at \url{www.ironkey.com}}
  \label{fig:ik}
\end{marginfigure}



\section{System Description}
\label{sec:sysdesc}

The password manager consists of the USB drive, a webserver for software and
firmware updates and an embedded password manager app.  The password manager is
accessed through the USB drive which must first be unlocked via a host machine
running Windows, Mac or Linux operating systems. The USB drive contains an
encrypted partition and an encrypted partition.  The encrypted partition is
formed on USB initialization and uses a unique random key inside of the USB to
generate a second random key which is used to create the encrypted
partition. These two keys are embedded in the hardware and never shared with the
host OS.\sidenote{The device unique identifier and private key are contained
within the silicon of the device and can only be accesed through an
authenticated secure session application hosted on the USB drive.}  The USB also
provides cryptographic functions via a cryptochip inside the USB.  The USB also
provides a unique identifier which is contained within its embedded circuitry.
\par Once the USB drive is inserted into the host computer, a program is
executed from the unencrypted partition which mounts the USB to the host OS.
Following mounting, a small program from the encrypted partition is executed
which prompts the user for a password. After a preset number of failed password
attempts, the USB device will wipe all keys and execute additional functions to
render the USB device useless and unrecoverable.
\par The USB device firmware and software can be updated via a secure Internet
channel which accesses a webserver to download the device files to the USB
device.  Following download, the device authenticates the downloaded files and
updates the device.

\begin{marginfigure}
    \centering
    \includegraphics{DFD_Main}
    \caption{Main System Data Flow Diagram}
    \label{fig:dfd}
\end{marginfigure}
\begin{marginfigure}
    \centering
    \includegraphics{DFD_USB}
    \caption{System Data Flow Diagram in the USB side}
    \label{fig:dfd}
\end{marginfigure}
\begin{marginfigure}
    \centering
    \includegraphics{DFD_App}
    \caption{System Data Flow Diagram in the App side}
    \label{fig:dfd}
\end{marginfigure}

The system data flow diagram is shown in figure \ref{fig:dfd}

\section{Functional Requirements}
\label{sec:funcreq}
Describe the functions of the system.  In here place swim lane
diagrams corresponding to each major functional block details in
\ref{sec:sysdesc}

\section{Security Requirements}
\label{sec:secreqs}
Following the example in Shostack~\cite{shostackbook}, we describe the
security requirements of the system which will inform our threat
model.
