\chapter{System Overview}
\label{ch:System Overview}
The system modeled is a web-enabled password manager that resides on a
tamper resistant USB thumb drive, modelled after the
IronKey\texttrademark secure USB thumb drive. The password manager
allows for a user to enjoy the benefits of a secure offline password
manager with the option of storing information to the cloud.
Additionally, the benefits of a embedded crypto system provide
robustness to the user.  The device firmware can be updated via an
internet connection.

\begin{marginfigure}%
\centering
  \includegraphics[width=0.25\linewidth]{s1000-vertical}
  \caption{Picture of the IronKey USB drive.  More information can be
found at \url{www.ironkey.com}}
  \label{fig:ik}
\end{marginfigure}



\section{System Description}
\label{sec:sysdesc}

\pat{this is a comment by pat}\\
\donginn{this is a comment by DongInn}\\
\john{this is a comment by John}\\
\akshada{this is a comment by Akshada}\\


The offline portion of the password manager consists of the USB drive
and a host machine running Windows, Mac or Linux operating
systems. The USB drive contains an encrypted partition and an
encrypted partition.  The encrypted partition is formed on USB
initialization and uses a unique random key inside of the USB to
generate a second random key which is used to create the encrypted
partition. These two keys are embedded in the hardware and never
shared with the host OS.
\par Once the USB drive is inserted into the host computer, a program
is executed from the unencrypted partition which mounts the USB to the
host OS.  Following mounting, a small program from the encrypted
partition is executed which prompts the user for a password. After a
preset number of failed password attempts, the USB device will wipe
all keys and execute additional functions to render the USB device
useless and unrecoverable.\sidenote{\pat{the sidenote command can be
used to highlight a finding or concept}}
\par Web password functions are optionally enabled and use the USB
drive as the seed key to set up secure connections to a server which
will host the user passwords.  The server has no access to the USB
keys (they are written to either flash or silicon) but instead sets up
a separate password environment which allows segregation of passwords
and keys, allowing a user to choose where to store passwords and keys.

\begin{figure}[h]
    \centering
    \includegraphics[width=0.75\linewidth]{State_Diagram}
    \caption{Password Manager State Diagram for Offline Usage}
    \label{fig:state_diagram}
\end{figure}

\section{Functional Requirements}
\label{sec:funcreq}
Describe the functions of the system.  In here place swim lane
diagrams corresponding to each major functional block details in
\ref{sec:sysdesc}

\section{Security Requirements}
\label{sec:secreqs}
Following the example in Shostack~\cite{shostackbook}, we describe the
security requirements of the system which will inform our threat
model.
